% !TEX program = pdflatex
% !TEX spellcheck = en_US
\documentclass[aspectratio=169,svgnames,xcolor=table,t]{beamer}
%%%%%%%%%%%%%%%%%%%%%%%%%%%%%%%%%%%%%%%%%%%%%%%%%%%%%%%%%%%%%%%%%%%%%%
\usepackage{listings}
\lstset{basicstyle=\ttfamily,escapechar=\%,escapeinside={\#(}{\#)}}
%%%%%%%%%%%%%%%%%%%%%%%%%%%%%%%%%%%%%%%%%%%%%%%%%%%%%%%%%%%%%%%%%%%%%%
\usepackage{amsmath,amssymb,amstext}
\usepackage{mathtools}
\usepackage{calc}
\usepackage{tikz}
\usetikzlibrary{positioning}
\tikzset{place/.style={draw}}
\tikzset{>=stealth, auto, node distance=2.5cm, every loop/.style={->, min distance=10mm, in=0, out=60, looseness=10}}
\usepackage{pgfplots}
\usepackage{algorithm, algorithmic}
%%%%%%%%%%%%%%%%%%%%%%%%%%%%%%%%%%%%%%%%%%%%%%%%%%%%%%%%%%%%%%%%%%%%%%
% Font Settings

\usepackage{anyfontsize}
\usepackage[utf8]{inputenc}
\usepackage[T1]{fontenc}
\usepackage[lighttt]{lmodern}
\usepackage[sfdefault,scaled=0.85,lf]{FiraSans}
\usepackage[scaled=0.85]{FiraMono}
\usepackage[scaled=0.85]{newtxsf}
\renewcommand\rmdefault{\sfdefault}
\usepackage[english]{babel}
\usepackage[autostyle,english=american]{csquotes}
\usepackage[protrusion,expansion,kerning,spacing,final]{microtype}
\microtypecontext{spacing=nonfrench}

% Color Definitions

\definecolor{SB}{RGB}{89,107,131}
\definecolor{SR}{RGB}{132,67,58}
\definecolor{SG}{RGB}{62,97,27}
\definecolor{CH}{RGB}{85,85,85}

% Beamer Setup

\usetheme{metropolis}
\metroset{subsectionpage=progressbar}
\makeatletter
\setbeamertemplate{section page}{
  \centering
  \begin{minipage}{0.8\linewidth}
    \raggedright
    \usebeamercolor[fg]{section title}
    \usebeamerfont{section title}
    \insertsectionhead\\[-1ex]
    \usebeamertemplate*{progress bar in section page}
    \par
    \ifx\insertsubsectionhead\@empty\else%
      \usebeamercolor[fg]{subsection title}%
      \usebeamerfont{subsection title}%
      \insertsubsectionhead
    \fi
  \end{minipage}
  \par
  \vspace{\baselineskip}
}
\newcommand{\disablesectionpage}{\metropolis@disablesectionpage}
\newcommand{\enablesectionpage}{\metropolis@enablesectionpage}
\makeatother


\setbeamercolor{frametitle}{bg=black}
\setbeamercolor{progress bar}{fg=SR,bg=SR}
\setbeamercolor{title separator}{fg=SR,bg=SR}
\setbeamercolor{structure}{fg=gray}
\setbeamerfont{structure}{series=\bfseries}


% Theorem style

\makeatletter
\renewcommand<>\beamer@begintheorem[1][]{%
    \ifblank{#1}
    {\def\inserttheoremaddition{}}
    {\def\inserttheoremaddition{#1}}
    \usebeamertemplate{theorem begin}#2\vspace{1pt}}
  \def\@endtheorem{\usebeamertemplate{theorem end}\vspace{1pt}}
\makeatother

\newenvironment{myitemize}{%
\setlength{\leftmargini}{0pt}%
\begin{itemize}%
\setlength{\itemsep}{5pt}%
}{%
\end{itemize}%
}

\newenvironment{mmyitemize}{%
\begin{itemize}%
\setlength{\itemsep}{5pt}
}{%
\end{itemize}%
}

\newcommand{\hi}[1]{\alert{#1}}

\let\oldtitlepage\titlepage

\renewcommand{\titlepage}{%
\let\oldinserttitle\inserttitle%
\renewcommand\inserttitle{\oldinserttitle\titlecovernote}%
\oldtitlepage%
\let\inserttitle\oldinserttitle%
}


\newcommand\backcoverpage{%
\let\oldinsertinstitute\insertinstitute%
\renewcommand\insertinstitute{\alert{\Large\bf Thanks for your attention!}}%
\oldtitlepage%
\let\insertinstitute\oldinsertinstitute%
}


\let\titlepage\oldtitlepage


%%%%%%%%%%%%%%%%%%%%%%%%%%%%%%%%%%%%%%%%%%%%%%%%%%%%%%%%%%%%%%%%%%

% \newtheorem{theorem}{Theorem}
% \newtheorem*{mtheorem}{Main Theorem}
% \newtheorem{lemma}{Lemma}
% \newtheorem{proposition}{Proposition}
% \newtheorem{definition}{Definition}
% \newtheorem{corollary}{Corollary}
% \newtheorem{example}{Example}
% \newtheorem{observation}{Observation}

\newcommand{\K}{\ensuremath{\mathbf{K}}}
\newcommand{\M}{\ensuremath{\mathbf{M}}}
\newcommand{\nott}[1]{\ensuremath{\mathit{n#1}}}
\newcommand{\notnot}[1]{\ensuremath{\mathit{nn#1}}}
\newcommand{\kk}[1]{\ensuremath{\mathit{k#1}}}
\newcommand{\kknott}[1]{\ensuremath{\mathit{kn#1}}}
\newcommand{\kp}[1]{\ensuremath{\mathit{pk#1}}}
\newcommand{\kpnott}[1]{\ensuremath{\mathit{pkn#1}}}
\newcommand{\kpn}[1]{\ensuremath{\mathit{nk#1}}}
\newcommand{\NF}[1]{\ensuremath{\mathit{NF}(#1)}}
\newcommand{\AT}{\ensuremath{\mathit{At}}}
\newcommand{\At}[1]{\ensuremath{\mathit{At}(#1)}}
\newcommand{\Atk}[1]{\ensuremath{\mathit{At^k}(#1)}}
\newcommand{\restr}[2]{{#1}_{\!|\!#2}}
\newcommand{\SM}[1]{\text{\rm SM}[#1]}
\newcommand{\SMC}[1]{\text{\rm CC}[#1]}
\newcommand{\SMCA}[1]{\text{\rm ACC}[#1]}
\newcommand{\Lit}{\ensuremath{\mathit{Lit}}}

\newcommand\wv{\mathbb{W}}

\newcommand{\EPASP}{\texttt{EP-ASP}}
\newcommand{\eclingo}{\texttt{eclingo}}


\newcounter{programcounter}
\newcommand{\newprogramph}{\refstepcounter{programcounter}}
\newcommand{\newprogram}{\refstepcounter{programcounter}\ensuremath{\Pi_{\theprogramcounter}}}
\newcommand{\program}[1]{\ensuremath{\Pi_{#1}}}
\newcommand{\speedup}[1]{\raisebox{0.1em}{\scriptsize$\sim$}#1}
%%%%%%%%%%%%%%%%%%%%%%%%%%%%%%%%%%%%%%%%%%%%%%%%%%%%%%%%%%%%%%%%%%%%%%
% Use enumitem and restore beamer defaults
% \usepackage{enumitem}
% \setitemize{label=\usebeamerfont*{itemize item}%
%   \usebeamercolor[fg]{itemize item}
%   \usebeamertemplate{itemize item}}
%%%%%%%%%%%%%%%%%%%%%%%%%%%%%%%%%%%%%%%%%%%%%%%%%%%%%%%%%%%%%%%%%%%%%%
\author[J. Fandinno \& Eleuterio Lillo Portero]{{\usebeamercolor[fg]{alerted text}Jorge Fandinno} and Eleuterio Lillo Portero}

\date{AAAI 2025}

\institute[UNO]{
    Department of Computer Science \\
    University of Nebraska Omaha
}
%%%%%%%%%%%%%%%%%%%%%%%%%%%%%%%%%%%%%%%%%%%%%%%%%%%%%%%%%%%%%%%%%%%%%%
%%%%%%%%%%%%%%%%%%%%%%%%%%%%%%%%%%%%%%%%%%%%%%%%%%%%%%%%%%%%%%%%%%%%%%
% Document Metadata
\title{Solving Epistemic Logic Programs using\\Generate-and-Test with Propagation}
%%%%%%%%%%%%%%%%%%%%%%%%%%%%%%%%%%%%%%%%%%%%%%%%%%%%%%%%%%%%%%%%%%%%%%
\begin{document}
\frame{\titlepage}
%%%%%%%%%%%%%%%%%%%%%%%%%%%%%%%%%%%%%%%%%%%%%%%%%%%%%%%%%%%%%%%%%%%%%%
\begin{frame}{Epistemic Logic Programs}
    \begin{myitemize}
        \item Epistemic logic programs are an extension of Answer Set Programs (ASP) with epistemic operators that allow reasoning about the knowledge of agents. For example,
        \begin{align*}
            \mathit{felon}   &\leftarrow \K\, \mathit{break\_rule}
            \\
            \mathit{suspect} &\leftarrow \neg \K\, \mathit{break\_rule}, \neg \K \neg \mathit{break\_rule}
        \end{align*}
        This encodes that a person a~$\mathit{felon}$ if we can determine that she broke a rule, and she is a~$\mathit{suspect}$ if it cannot be determined that she has broken nor that she has not done.
    
        \bigskip

        \item Deciding if an epistemic logic program has an answer (called \hi{worldview}) is~$\Sigma^P_3$-complete and~$\Sigma^P_2$-complete for the case of disjuntive\nobreak-free programs (One lever higher than ASP).
        
        \bigskip

        \item Epistemic logic programs provide a more natural way to model problem in the second level of the polynomial hierarchy than ASP with saturation techniques.
    \end{myitemize}
    \end{frame}
%%%%%%%%%%%%%%%%%%%%%%%%%%%%%%%%%%%%%%%%%%%%%%%%%%%%%%%%%%%%%%%%%%%%%%
\begin{frame}{In the paper}
    \begin{myitemize}
        \item We formalize an \hi{abstract generate-and-test framework} for computing worldviews that covers existing solvers such as \texttt{EP-ASP} or \texttt{eclingo}:
        \begin{itemize}
            \item Generate-and-test approaches use a generator program to generate candidates and a test program to check if a candidate corresponds to a worldview.

            \item We provide \hi{sufficient conditions} on the generator and test programs for the \hi{correctness} of the algorithm.
        \end{itemize}
        
        \medskip

        \item We propose a new generator program that \hi{propagates the consequences of epistemic decisions to reduce the number of candidates} that need to be tested.
        \begin{itemize}
            \item Each candidate to be tested requires a linear number of calls to an ASP solver.
        \end{itemize}

        \medskip

        \item We implement a new solver based on this new generator
        \begin{itemize}
            \item benchmarks show that it solves \hi{87\% more instances} than existing solvers and is \hi{\speedup{3.3x} faster}.
        \end{itemize}
    \end{myitemize}
    \end{frame}
% %%%%%%%%%%%%%%%%%%%%%%%%%%%%%%%%%%%%%%%%%%%%%%%%%%%%%%%%%%%%%%%%%%%%%%
% \begin{frame}{Solving Epistemic Logic Programs}
% \begin{myitemize}
%     \item \hi{\texttt{eclingo}} is a solver for epistemic logic programs.
%     \begin{align*}
%         \mathit{felon}   &\leftarrow \K\, \mathit{break\_rule}
%         \\
%         \mathit{suspect} &\leftarrow \neg \K\, \mathit{break\_rule}, \neg \K \neg \mathit{break\_rule}
%     \end{align*}
%     This encodes that a person a~$\mathit{felon}$ if we can determine that she broke a rule, and she is a~$\mathit{suspect}$ if it cannot be determined that she has broken nor that she has not done.

%     \item Deciding if an epistemic logic program has an answer (called \hi{worldview}) is~$\Sigma^P_3$-complete and~$\Sigma^P_2$-complete for the case of disjuntive\nobreak-free programs.
    
%     \item We developed a new version of \texttt{eclingo} that is \speedup{3.3x} faster and solves 87\% more instances.
% \end{myitemize}
% \end{frame}
% %%%%%%%%%%%%%%%%%%%%%%%%%%%%%%%%%%%%%%%%%%%%%%%%%%%%%%%%%%%%%%%%%%%%%%
% \begin{frame}{Solving Epistemic Logic Programs}
%     \begin{myitemize}
%         \item[] As the previous version, \texttt{eclingo} 2.0 uses a generate-and-test approach.
%     \end{myitemize}
%     \begin{algorithm}[H]
%         \scriptsize
%         \caption{Generate-and-test computation of~$n$ worldviews of a program~$\Pi$ in normal form.}\label{alg:guess-and-check}
%         \hspace*{\algorithmicindent} \textbf{Input} Generate program $G(\Pi)$\\
%         \hspace*{\algorithmicindent} \textbf{Input} Test program $T(\Pi)$\\
%         \hspace*{\algorithmicindent} \textbf{Input} Number of requested worldviews~$n$\\
%         \hspace*{\algorithmicindent} \textbf{Output} Set~$\Omega$ containing~$n$  worldviews of~$\Pi$
    
%         \begin{algorithmic}[1] %[1] enables line numbers
%             \STATE Let $\Omega=\emptyset$.
%             \FOR{$M$ in $\SM{G(\Pi)}$} \label{alg.generator.loop.ini}
%                 \IF {\texttt{Test$(T(\Pi),\,M)$}}
%                 \STATE $\wv$ = \texttt{BuildWorldView$(M)$}
%                 \STATE $\Omega$ = $\Omega \cup \{ \wv \}$
%                     \IF {$|\Omega| \geq n$}
%                     \STATE \textbf{return} $\Omega$
%                     \ENDIF
%                 \ENDIF
%             \ENDFOR \label{alg.generator.loop.end}
%             \STATE \textbf{return} $\Omega$
%         \end{algorithmic}
%     \end{algorithm}
% \end{frame}
% %%%%%%%%%%%%%%%%%%%%%%%%%%%%%%%%%%%%%%%%%%%%%%%%%%%%%%%%%%%%%%%%%%%%%%
\begin{frame}{Normal Form and Basic Generate Program}
    \begin{myitemize}
        \item \hi{Normal form}: We first compute a normal form of the program in which negation is not in the scope of the epistemic operator using auxiliary atoms.
        
        \item \hi{Basic Generate Program}: Replaces epistemic atoms~$\K\,a$ by a fresth atom~$\kk{a}$ and introduces a choice rule~$\{ \kk{a} \}$ for each fresh atom.
    \end{myitemize}
    Epistemic program
    \begin{align*}
        a && 
        b &\leftarrow \K a
        &
        c &\leftarrow \K b
    \end{align*}
    has a unique worldview~$[\{a,b,c\}]$. 
    \pause
    We get the generate program
    \begin{align*}
        a 
        && 
        b &\leftarrow ka
        &
        c &\leftarrow kb
        &
        \{ \kk{a} \} &\leftarrow a
        &
        \{ \kk{b} \} &\leftarrow b
    \end{align*}
    which has candidates~$\{a\}$, $\{a,\kk{a},b\}$ and~$\{a,\kk{a},b,\kk{b},c\}$. Only the last candidate corresponds to its unique worldview.
\end{frame}
%%%%%%%%%%%%%%%%%%%%%%%%%%%%%%%%%%%%%%%%%%%%%%%%%%%%%%%%%%%%%%%%%%%%%%
\begin{frame}{Propagate Epistemic Consequences}
    \vspace{-20pt}
    \begin{align*}
        a &&
        b &\leftarrow \K a
        &
        c &\leftarrow \K b
    \end{align*}
    To see that~$[\{a,b,c\}]$ is the unique worldview, we can reason as follows:
    \begin{itemize}
        \item $\K a$ is true because~$a$ is a fact.
        \item From~$\K a$ we can derive~$b$ and also~$\K b$.
        \item From~$\K b$ we can derive~$c$.
    \end{itemize}
    
    \pause
    \begin{myitemize}
        \item The key idea is to propagate the consequences of epistemic decisions.
        \item We can add rules that reflect this reasoning to the generator program.
    \end{myitemize}
    \begin{align*}
        a 
        && 
        b &\leftarrow ka
        &
        c &\leftarrow kb
        &
        \{ ka \} &\leftarrow a
        &
        \{ kb \} &\leftarrow b
        \\
        \hi{\kp{a}}&
        &
        \hi{\kp{b}} &\hi{\leftarrow \kk{a}}
        &
        \hi{\kp{c}} &\hi{\leftarrow \kk{b}}
        &
        \mathrlap{\hspace{-16pt}\hi{\leftarrow \kp{a} \wedge \neg ka}}
        &&
        \mathrlap{\hspace{-16pt}\hi{\leftarrow \kp{b} \wedge \neg \kk{b}}}
    \end{align*}
    \hi{$\kp{b}$} is used to distinguish that~$\K b$ is a consequence dervided from choices of epistemic atoms while~$kb$ only states that~$\K b$ has been chosen to be true.
\end{frame}
%%%%%%%%%%%%%%%%%%%%%%%%%%%%%%%%%%%%%%%%%%%%%%%%%%%%%%%%%%%%%%%%%%%%%%
\begin{frame}{Propagate Epistemic Consequences}
    \begin{myitemize}
        \item[] For every normal rule of the form
        \begin{align}
            a_1 &\leftarrow 
            a_2,\!\dotsc\!,a_k,
            \neg a_{k+1},\!\dotsc\!,\neg a_m,
            {L_{m+1}},\!\dotsc\!,{L_n}
        \end{align}
        with each~$L_i$ a~$\K$-atom, we add a rule of the form
        \begin{gather}
            \begin{aligned}
                \kp{a_1} &\span\span\leftarrow 
                \kp{a_2},\dotsc,\kp{a_k},
                \kpnott{a_{k+1}},\dotsc,\kpnott{a_m},
                \kk{L_{m+1}},\dotsc,\kk{L_n}
            \end{aligned}
        \end{gather}
        \vspace*{-15pt}
        \begin{itemize}
            \item \kp{a} means that we have concluded that~$a$ is true in all stable models corresponding to the considered candidate worldview.
            \item \kpnott{a} means that we have concluded that~$a$ is false in all stable models corresponding to the considered candidate worldview.
            \item atoms of the form~\kpnott{a} are derived through a kind of completion of the program.
        \end{itemize}
    \end{myitemize}
\end{frame}
%%%%%%%%%%%%%%%%%%%%%%%%%%%%%%%%%%%%%%%%%%%%%%%%%%%%%%%%%%%%%%%%%%%%%%
\begin{frame}{Propagate Epistemic Consequences}
    For program
    \begin{align*}
        b &\leftarrow \neg a
        &
        c &\leftarrow \K b
    \end{align*}
    we obtain the generator program
    \begin{align*}
        &&
        b &\leftarrow \neg a
        &
        c &\leftarrow kb
        &
        \{ \kk{b} \} &\leftarrow b
        \\
        \hi{\kpnott{a}}&
        &
        \kp{b} &\leftarrow \hi{\kpnott{a}}
        &
        \kp{c} &\leftarrow \kk{b}
        &&
        \mathrlap{\hspace{-16pt}\leftarrow \kp{b} \wedge \neg \kk{b}}
    \end{align*}
    which produces a single candidate~$\{\kp{a},b,\kp{b},\kk{b},c,\kp{c}\}$ that corresponds to the unique worldview~$[\{b,c\}]$.

    \begin{itemize}
        \item $\kpnott{a}$ is a fact because there is no rule with~$a$ in the head.
        \item The translation uses auxiliary atoms for each body to avoid the quadratic blowup of the completion.
    \end{itemize}
\end{frame}
%%%%%%%%%%%%%%%%%%%%%%%%%%%%%%%%%%%%%%%%%%%%%%%%%%%%%%%%%%%%%%%%%%%%%%
\begin{frame}[c]{Propagate Epistemic Consequences}
    \begin{theorem}
        This method produces a valid generator program, and, thus, it can be used in our generate-and-test framework to correctly compute the worldviews of a program.
        \\
        Furthermore,
        \begin{itemize}
            \item It produces at most as many candidates as the basic generator program.
            \item For some program classes, it produces exponentially fewer candidates.
            \item It only introduces a linear overhead.
        \end{itemize}
    \end{theorem}
\end{frame}
%%%%%%%%%%%%%%%%%%%%%%%%%%%%%%%%%%%%%%%%%%%%%%%%%%%%%%%%%%%%%%%%%%%%%%
\begin{frame}[c]{Implementation}
    \begin{myitemize}
        \item[] Our implementation is built on top of version~5.7 of the ASP solver~\texttt{clingo} using Python and ASP.%
        \begin{myitemize}
            \item It supports most language features of \texttt{clingo} such as choice rules, aggregates, intervals, etc.
            
            \item The code is available with MIT license at \hi{\url{https://github.com/potassco/eclingo}}.
            
            \item It replaces the previous version of eclingo.
                        
            \item We use reification and metaprogramming to describe the generate and test programs.
            
            \item This will allow to develop new solvers with different generate and test programs by only modifying the ASP metaprogram.
        \end{myitemize}
    \end{myitemize}
\end{frame}
%%%%%%%%%%%%%%%%%%%%%%%%%%%%%%%%%%%%%%%%%%%%%%%%%%%%%%%%%%%%%%%%%%%%%%
\begin{frame}{Experimental Results and Survival Plot}
    \thispagestyle{empty} 
    \begin{columns}[T]
        \begin{column}{0.45\textwidth}
            \begin{myitemize}
                \small
                \item It solves 87\% more instances than existing solvers and is \speedup{3.3x} faster.

                \bigskip

                \item It dominates the other solvers in all benchmarks.
                
                \bigskip
                
                \item \texttt{EP-ASP} has a specific mode for conformant planning that uses domain-specific heuristics.
                \begin{itemize}
                    \item Our solver is \speedup{1.5x} faster and solves \speedup{50\%} more instances despite not using domain-specific heuristics.
                \end{itemize}
            \end{myitemize}
        \end{column}
        \begin{column}{0.5\textwidth}
            \hspace*{-15pt}
            %!TEX root = paper-aaai25.tex
\begin{tikzpicture}
\pgfplotsset{every tick label/.append style={font=\scriptsize}}
\begin{axis}[
    % title={Survival Plots for solvers},
    width=1.25\textwidth,
    height=1.13\textwidth,
    ylabel={\scriptsize Time(Seconds)},
    xlabel={\scriptsize Nº of Instances solved},
    ylabel style={
        yshift=-5pt,
        % xshift=-60pt,
    },
    ymin=0, ymax=603,
    xmin=0, xmax=350,
    ytick={0, 60, 120, 180, 240, 300, 360, 420, 480, 540, 600},
    xtick={50, 100, 150, 200, 250, 300, 350},
    legend style={
                at={(0.87,0.165)},
                anchor=center,
                legend columns=1,
                nodes={scale=0.6, transform shape},
                /tikz/every even column/.append style={column sep=0.5cm}
            },
    ymajorgrids=true,
    grid style=dashed,
]
\addplot[
    color=blue,
    mark=*,
    ]
    coordinates {
    (1,0.154922)(2,0.16203)(3,0.16743999999999998)(4,0.1717455)(5,0.1776755)(6,0.18612099999999998)(7,0.192922)(8,0.192998)(9,0.194574)(10,0.1951465)(11,0.2026765)(12,0.20275949999999998)(13,0.2119715)(14,0.2133575)(15,0.215201)(16,0.21608899999999998)(17,0.219945)(18,0.2219755)(19,0.229006)(20,0.2316645)(21,0.235128)(22,0.2425365)(23,0.2457745)(24,0.252011)(25,0.25256850000000003)(26,0.254865)(27,0.255891)(28,0.2610625)(29,0.2653535)(30,0.2660585)(31,0.26918600000000004)(32,0.30066000000000004)(33,0.30200400000000005)(34,0.3026175)(35,0.3041345)(36,0.3067565)(37,0.30746700000000005)(38,0.31862)(39,0.32464899999999997)(40,0.3256645)(41,0.3263325)(42,0.3655485)(43,0.38409899999999997)(44,0.4118465)(45,0.425622)(46,0.430578)(47,0.43336600000000003)(48,0.448606)(49,0.45592299999999997)(50,0.483968)(51,0.4881005)(52,0.4889105)(53,0.5045935)(54,0.5082450000000001)(55,0.5188925)(56,0.5458745)(57,0.560948)(58,0.5641455)(59,0.6128235)(60,0.618761)(61,0.6312105)(62,0.6723705)(63,0.6821665)(64,0.69794)(65,0.711919)(66,0.721591)(67,0.7423455)(68,0.7961125)(69,0.7980195)(70,0.808441)(71,0.836962)(72,0.839183)(73,0.933079)(74,0.942664)(75,0.9718495)(76,0.97444)(77,0.9779344999999999)(78,0.991527)(79,0.9930885)(80,1.0209700000000002)(81,1.0946449999999999)(82,1.10531)(83,1.15149)(84,1.1866349999999999)(85,1.210845)(86,1.23458)(87,1.2388750000000002)(88,1.31433)(89,1.421795)(90,1.43323)(91,1.444695)(92,1.48672)(93,1.550775)(94,1.551265)(95,1.55156)(96,1.669785)(97,1.8561999999999999)(98,1.976875)(99,2.00042)(100,2.0389049999999997)(101,2.096415)(102,2.186255)(103,2.22913)(104,2.315905)(105,2.3898650000000004)(106,2.54013)(107,2.60611)(108,2.60944)(109,2.6443899999999996)(110,2.741485)(111,2.840595)(112,3.007345)(113,3.084695)(114,3.208935)(115,3.2174750000000003)(116,3.4957000000000003)(117,3.6079600000000003)(118,3.836605)(119,3.87767)(120,4.023855)(121,4.22523)(122,4.327075)(123,4.50767)(124,4.746745)(125,4.75872)(126,4.82061)(127,5.062799999999999)(128,5.2063950000000006)(129,5.473305)(130,5.647815)(131,5.750845)(132,5.927935)(133,5.9856)(134,6.01107)(135,6.25594)(136,6.6053999999999995)(137,6.780865)(138,7.45555)(139,7.601205)(140,8.3549)(141,8.50667)(142,8.93148)(143,9.75158)(144,9.891905)(145,10.452649999999998)(146,11.08155)(147,11.118549999999999)(148,11.83055)(149,12.226099999999999)(150,12.7446)(151,12.918099999999999)(152,13.8072)(153,14.99855)(154,15.16465)(155,15.1822)(156,15.21805)(157,15.637550000000001)(158,17.998649999999998)(159,18.0412)(160,18.66775)(161,20.0746)(162,20.2927)(163,21.0214)(164,21.14295)(165,21.7038)(166,22.141350000000003)(167,23.45805)(168,24.112450000000003)(169,25.30105)(170,26.61605)(171,26.9514)(172,28.98065)(173,30.1477)(174,32.513000000000005)(175,34.3553)(176,34.602850000000004)(177,35.5378)(178,37.0947)(179,38.03035)(180,40.04715)(181,41.0664)(182,42.72505)(183,43.044)(184,46.64125)(185,47.21875)(186,47.9634)(187,48.93729999999999)(188,49.8788)(189,51.02385)(190,51.62195)(191,52.5173)(192,53.80455)(193,54.90755)(194,55.09935)(195,55.6484)(196,61.11785)(197,64.0018)(198,67.27529999999999)(199,68.43615)(200,68.97585000000001)(201,71.16885)(202,72.803)(203,74.79355)(204,76.7559)(205,78.85485)(206,79.62270000000001)(207,84.57554999999999)(208,85.4471)(209,89.161)(210,91.00555)(211,91.6671)(212,97.90975)(213,99.47255)(214,99.60055)(215,102.1345)(216,102.7355)(217,106.424)(218,107.0495)(219,108.18)(220,111.311)(221,113.22)(222,117.193)(223,117.432)(224,118.82)(225,119.5865)(226,119.605)(227,123.08000000000001)(228,128.4465)(229,129.745)(230,130.29899999999998)(231,133.5515)(232,136.2095)(233,137.7775)(234,143.461)(235,147.75549999999998)(236,147.81400000000002)(237,150.659)(238,151.886)(239,157.6635)(240,160.24149999999997)(241,164.90050000000002)(242,169.23950000000002)(243,170.856)(244,172.566)(245,177.9085)(246,181.6165)(247,184.4385)(248,194.03199999999998)(249,196.68200000000002)(250,198.5965)(251,205.7255)(252,207.32850000000002)(253,211.034)(254,213.25900000000001)(255,214.421)(256,214.50549999999998)(257,219.112)(258,223.61700000000002)(259,225.3065)(260,227.43)(261,233.64749999999998)(262,236.6765)(263,238.001)(264,238.57150000000001)(265,243.904)(266,244.97250000000003)(267,245.851)(268,251.5865)(269,255.33)(270,256.7405)(271,256.754)(272,258.7155)(273,269.6885)(274,273.637)(275,280.01)(276,280.3825)(277,282.85900000000004)(278,294.4145)(279,299.4505)(280,307.072)(281,317.73900000000003)(282,321.061)(283,337.3365)(284,352.3805)(285,357.814)(286,367.0725)(287,381.804)(288,394.17)(289,441.2795)(290,530.3815)(291,545.6645000000001)(292,600.0)(293,600.0)(294,600.0)(295,600.0)(296,600.0)(297,600.0)(298,600.0)(299,600.0)(300,600.0)(301,600.0)(302,600.0)(303,600.0)(304,600.0)(305,600.0)(306,600.0)(307,600.0)(308,600.0)(309,600.0)(310,600.0)(311,600.0)(312,600.0)(313,600.0)(314,600.0)(315,600.0)(316,600.0)(317,600.0)(318,600.0)(319,600.0)(320,600.0)(321,600.0)(322,600.0)(323,600.0)(324,600.0)(325,600.0)(326,600.0)(327,600.0)(328,600.0)(329,600.0)(330,600.0)(331,600.0)(332,600.0)(333,600.0)
};
\addlegendentry{ new solver }
% \addplot[
    color=red,
    mark=x,
    ]
    coordinates {
    (1,0.15041)(2,0.1695065)(3,0.1763245)(4,0.177056)(5,0.178106)(6,0.185884)(7,0.18772650000000002)(8,0.191318)(9,0.191753)(10,0.19479400000000002)(11,0.1952105)(12,0.20246799999999998)(13,0.202577)(14,0.2073885)(15,0.2115565)(16,0.2135805)(17,0.220354)(18,0.222414)(19,0.223356)(20,0.224848)(21,0.2252825)(22,0.22741050000000002)(23,0.22871)(24,0.239037)(25,0.24079299999999998)(26,0.2421605)(27,0.247023)(28,0.254274)(29,0.26283599999999996)(30,0.26360700000000004)(31,0.265543)(32,0.2664995)(33,0.2736325)(34,0.274786)(35,0.27784)(36,0.283919)(37,0.2869875)(38,0.290599)(39,0.292231)(40,0.297362)(41,0.318328)(42,0.3189695)(43,0.3342115)(44,0.348418)(45,0.4074955)(46,0.4137625)(47,0.429988)(48,0.43520250000000005)(49,0.463785)(50,0.466029)(51,0.47113150000000004)(52,0.47361200000000003)(53,0.4900185)(54,0.499)(55,0.5178205)(56,0.542673)(57,0.5681285)(58,0.5966265)(59,0.6486845)(60,0.6715329999999999)(61,0.688707)(62,0.7081155)(63,0.7155294999999999)(64,0.7233815)(65,0.7304685)(66,0.7634765)(67,0.7974915)(68,0.826448)(69,0.844858)(70,0.8781384999999999)(71,0.912397)(72,0.9474035000000001)(73,0.989197)(74,1.11036)(75,1.120935)(76,1.178665)(77,1.18595)(78,1.1964000000000001)(79,1.24127)(80,1.368805)(81,1.371175)(82,1.400075)(83,1.4752649999999998)(84,1.509625)(85,1.568675)(86,1.5688300000000002)(87,1.77525)(88,1.920955)(89,1.98203)(90,2.07602)(91,2.092605)(92,2.1008250000000004)(93,2.197965)(94,2.430505)(95,2.52485)(96,2.52959)(97,2.7132500000000004)(98,2.8306750000000003)(99,2.938875)(100,3.02799)(101,3.0765450000000003)(102,3.34694)(103,3.668025)(104,3.712275)(105,3.896785)(106,4.317785)(107,4.79964)(108,5.025)(109,5.161300000000001)(110,5.23299)(111,5.443614999999999)(112,5.658175)(113,5.901265)(114,6.09797)(115,6.521515)(116,6.648945)(117,6.940085)(118,7.348135)(119,7.97923)(120,8.27866)(121,8.663575)(122,9.45281)(123,10.620149999999999)(124,10.653099999999998)(125,11.5698)(126,11.86205)(127,12.3448)(128,12.38325)(129,13.03425)(130,13.8794)(131,14.4361)(132,14.75665)(133,16.9171)(134,17.805799999999998)(135,18.02885)(136,18.203049999999998)(137,19.158050000000003)(138,20.9824)(139,21.2733)(140,23.1934)(141,26.39025)(142,28.223399999999998)(143,29.2519)(144,31.139499999999998)(145,34.03475)(146,36.41635)(147,38.102149999999995)(148,41.229150000000004)(149,44.988550000000004)(150,46.33755)(151,50.3845)(152,50.8889)(153,50.975300000000004)(154,52.2868)(155,58.18495)(156,60.73415)(157,64.63325)(158,71.7572)(159,75.418)(160,79.31434999999999)(161,83.53710000000001)(162,86.70779999999999)(163,89.2333)(164,97.7234)(165,102.04050000000001)(166,102.239)(167,108.3335)(168,111.038)(169,111.501)(170,117.26650000000001)(171,136.49849999999998)(172,251.95999999999998)(173,289.62699999999995)(174,551.5305)(175,600.0)(176,600.0)(177,600.0)(178,600.0)(179,600.0)(180,600.0)(181,600.0)(182,600.0)(183,600.0)(184,600.0)(185,600.0)(186,600.0)(187,600.0)(188,600.0)(189,600.0)(190,600.0)(191,600.0)(192,600.0)(193,600.0)(194,600.0)(195,600.0)(196,600.0)(197,600.0)(198,600.0)(199,600.0)(200,600.0)(201,600.0)(202,600.0)(203,600.0)(204,600.0)(205,600.0)(206,600.0)(207,600.0)(208,600.0)(209,600.0)(210,600.0)(211,600.0)(212,600.0)(213,600.0)(214,600.0)(215,600.0)(216,600.0)(217,600.0)(218,600.0)(219,600.0)(220,600.0)(221,600.0)(222,600.0)(223,600.0)(224,600.0)(225,600.0)(226,600.0)(227,600.0)(228,600.0)(229,600.0)(230,600.0)(231,600.0)(232,600.0)(233,600.0)(234,600.0)(235,600.0)(236,600.0)(237,600.0)(238,600.0)(239,600.0)(240,600.0)(241,600.0)(242,600.0)(243,600.0)(244,600.0)(245,600.0)(246,600.0)(247,600.0)(248,600.0)(249,600.0)(250,600.0)(251,600.0)(252,600.0)(253,600.0)(254,600.0)(255,600.0)(256,600.0)(257,600.0)(258,600.0)(259,600.0)(260,600.0)(261,600.0)(262,600.0)(263,600.0)(264,600.0)(265,600.0)(266,600.0)(267,600.0)(268,600.0)(269,600.0)(270,600.0)(271,600.0)(272,600.0)(273,600.0)(274,600.0)(275,600.0)(276,600.0)(277,600.0)(278,600.0)(279,600.0)(280,600.0)(281,600.0)(282,600.0)(283,600.0)(284,600.0)(285,600.0)(286,600.0)(287,600.0)(288,600.0)(289,600.0)(290,600.0)(291,600.0)(292,600.0)(293,600.0)(294,600.0)(295,600.0)(296,600.0)(297,600.0)(298,600.0)(299,600.0)(300,600.0)(301,600.0)(302,600.0)(303,600.0)(304,600.0)(305,600.0)(306,600.0)(307,600.0)(308,600.0)(309,600.0)(310,600.0)(311,600.0)(312,600.0)(313,600.0)(314,600.0)(315,600.0)(316,600.0)(317,600.0)(318,600.0)(319,600.0)(320,600.0)(321,600.0)(322,600.0)(323,600.0)(324,600.0)(325,600.0)(326,600.0)(327,600.0)(328,600.0)(329,600.0)(330,600.0)(331,600.0)(332,600.0)(333,600.0)
};
\addlegendentry{ $G_0$ }
\addplot[
    dashed,
    color=green,
    mark=*,
    ]
    coordinates {
    (1,0.082662)(2,0.0828535)(3,0.0831845)(4,0.0833445)(5,0.0833705)(6,0.0839715)(7,0.08457049999999999)(8,0.0848215)(9,0.084984)(10,0.086015)(11,0.0862875)(12,0.086501)(13,0.087004)(14,0.087201)(15,0.0874515)(16,0.087807)(17,0.0879065)(18,0.087922)(19,0.08793400000000001)(20,0.0880055)(21,0.088144)(22,0.08891399999999999)(23,0.0896905)(24,0.0902195)(25,0.0907735)(26,0.0909055)(27,0.0912965)(28,0.09182)(29,0.0935155)(30,0.0935485)(31,0.093977)(32,0.094224)(33,0.094467)(34,0.09461800000000001)(35,0.0947105)(36,0.09559300000000001)(37,0.0965305)(38,0.0974435)(39,0.09819249999999999)(40,0.098495)(41,0.10149749999999999)(42,0.10213900000000001)(43,0.10734199999999999)(44,0.107757)(45,0.108198)(46,0.110652)(47,0.116497)(48,0.11683750000000001)(49,0.12095700000000001)(50,0.1212935)(51,0.13688699999999998)(52,0.1375665)(53,0.14170549999999998)(54,0.15276450000000003)(55,0.1604825)(56,0.1667415)(57,0.218474)(58,0.235753)(59,0.2503865)(60,0.33779349999999997)(61,0.349786)(62,0.3843665)(63,0.4057265)(64,0.4113945)(65,0.43431949999999997)(66,0.4657545)(67,0.514817)(68,0.562808)(69,0.5675479999999999)(70,0.6242505)(71,0.713753)(72,0.8287015)(73,0.830163)(74,0.8623719999999999)(75,0.9288915)(76,0.9914480000000001)(77,1.1481599999999998)(78,1.7825199999999999)(79,1.915895)(80,1.91915)(81,1.926295)(82,2.0416)(83,2.11333)(84,2.116705)(85,2.3074250000000003)(86,2.38796)(87,2.651985)(88,3.386)(89,4.031855)(90,4.1553450000000005)(91,4.425585)(92,4.579845000000001)(93,5.63999)(94,5.696009999999999)(95,5.804365)(96,5.965145)(97,6.923445)(98,7.276)(99,7.694355)(100,7.774240000000001)(101,8.5379)(102,8.62839)(103,9.03755)(104,9.31514)(105,9.900110000000002)(106,10.4409)(107,11.04005)(108,11.1395)(109,11.66215)(110,15.418199999999999)(111,15.938399999999998)(112,16.6226)(113,18.34385)(114,19.83915)(115,22.9127)(116,23.274250000000002)(117,26.9637)(118,27.5066)(119,28.5473)(120,29.488500000000002)(121,30.3283)(122,31.64)(123,34.060199999999995)(124,43.73015)(125,54.560050000000004)(126,61.6233)(127,63.69584999999999)(128,78.60974999999999)(129,87.1776)(130,91.82005)(131,106.906)(132,107.232)(133,112.768)(134,141.7865)(135,152.887)(136,169.9865)(137,177.3405)(138,211.687)(139,222.51999999999998)(140,308.89)(141,336.8235)(142,352.12800000000004)(143,393.4995)(144,426.6805)(145,474.66999999999996)(146,484.2925)(147,501.985)(148,516.5160000000001)(149,578.5364999999999)(150,586.948)(151,588.4205)(152,591.1700000000001)(153,591.7785)(154,594.927)(155,595.6645)(156,595.745)(157,597.1179999999999)(158,597.578)(159,597.5835)(160,597.585)(161,598.3895)(162,598.5805)(163,598.6205)(164,598.9955)(165,599.2265)(166,599.4365)(167,599.694)(168,599.7065)(169,600.0)(170,600.0)(171,600.0)(172,600.0)(173,600.0)(174,600.0)(175,600.0)(176,600.0)(177,600.0)(178,600.0)(179,600.0)(180,600.0)(181,600.0)(182,600.0)(183,600.0)(184,600.0)(185,600.0)(186,600.0)(187,600.0)(188,600.0)(189,600.0)(190,600.0)(191,600.0)(192,600.0)(193,600.0)(194,600.0)(195,600.0)(196,600.0)(197,600.0)(198,600.0)(199,600.0)(200,600.0)(201,600.0)(202,600.0)(203,600.0)(204,600.0)(205,600.0)(206,600.0)(207,600.0)(208,600.0)(209,600.0)(210,600.0)(211,600.0)(212,600.0)(213,600.0)(214,600.0)(215,600.0)(216,600.0)(217,600.0)(218,600.0)(219,600.0)(220,600.0)(221,600.0)(222,600.0)(223,600.0)(224,600.0)(225,600.0)(226,600.0)(227,600.0)(228,600.0)(229,600.0)(230,600.0)(231,600.0)(232,600.0)(233,600.0)(234,600.0)(235,600.0)(236,600.0)(237,600.0)(238,600.0)(239,600.0)(240,600.0)(241,600.0)(242,600.0)(243,600.0)(244,600.0)(245,600.0)(246,600.0)(247,600.0)(248,600.0)(249,600.0)(250,600.0)(251,600.0)(252,600.0)(253,600.0)(254,600.0)(255,600.0)(256,600.0)(257,600.0)(258,600.0)(259,600.0)(260,600.0)(261,600.0)(262,600.0)(263,600.0)(264,600.0)(265,600.0)(266,600.0)(267,600.0)(268,600.0)(269,600.0)(270,600.0)(271,600.0)(272,600.0)(273,600.0)(274,600.0)(275,600.0)(276,600.0)(277,600.0)(278,600.0)(279,600.0)(280,600.0)(281,600.0)(282,600.0)(283,600.0)(284,600.0)(285,600.0)(286,600.0)(287,600.0)(288,600.0)(289,600.0)(290,600.0)(291,600.0)(292,600.0)(293,600.0)(294,600.0)(295,600.0)(296,600.0)(297,600.0)(298,600.0)(299,600.0)(300,600.0)(301,600.0)(302,600.0)(303,600.0)(304,600.0)(305,600.0)(306,600.0)(307,600.0)(308,600.0)(309,600.0)(310,600.0)(311,600.0)(312,600.0)(313,600.0)(314,600.0)(315,600.0)(316,600.0)(317,600.0)(318,600.0)(319,600.0)(320,600.0)(321,600.0)(322,600.0)(323,600.0)(324,600.0)(325,600.0)(326,600.0)(327,600.0)(328,600.0)(329,600.0)(330,600.0)(331,600.0)(332,600.0)(333,600.0)
    };
    \addlegendentry{ \texttt{eclingo} }
% \addplot[
    dashed,
    color=yellow,
    mark=x,
    ]
    coordinates {
    (1,0.001571)(2,0.008602499999999999)(3,0.009071)(4,0.012036999999999999)(5,0.015216)(6,0.0156585)(7,0.0156585)(8,0.0156585)(9,0.02972)(10,0.08504149999999999)(11,0.09093899999999999)(12,0.090964)(13,0.0915585)(14,0.09231600000000001)(15,0.09399550000000001)(16,0.09633749999999999)(17,0.097559)(18,0.10023)(19,0.10125)(20,0.1027975)(21,0.1073625)(22,0.10890050000000001)(23,0.1169065)(24,0.11887600000000001)(25,0.119216)(26,0.1234645)(27,0.129986)(28,0.138416)(29,0.140868)(30,0.14647949999999998)(31,0.15289799999999998)(32,0.153177)(33,0.1549975)(34,0.1613075)(35,0.18138100000000001)(36,0.182256)(37,0.185805)(38,0.1898235)(39,0.190066)(40,0.2918535)(41,0.3604)(42,0.3625795)(43,0.37163500000000005)(44,0.39266650000000003)(45,0.415846)(46,0.4331955)(47,0.503023)(48,0.5208124999999999)(49,0.659082)(50,0.668829)(51,0.6939915)(52,1.0067205000000001)(53,1.097325)(54,1.2383899999999999)(55,1.30443)(56,1.427545)(57,1.777155)(58,1.93245)(59,2.060285)(60,2.07011)(61,2.35268)(62,2.4309000000000003)(63,2.8895299999999997)(64,3.135085)(65,3.27035)(66,3.4832)(67,3.683225)(68,4.557295)(69,4.86781)(70,4.98204)(71,5.285905)(72,5.337695)(73,7.34757)(74,7.844234999999999)(75,8.480595000000001)(76,8.882615000000001)(77,10.42)(78,10.59535)(79,10.9185)(80,14.444600000000001)(81,15.455950000000001)(82,15.6878)(83,19.33175)(84,20.43385)(85,20.79965)(86,24.502299999999998)(87,24.86385)(88,25.2709)(89,27.154)(90,32.46835)(91,35.224199999999996)(92,36.6045)(93,37.29345)(94,37.52235)(95,40.977450000000005)(96,43.1949)(97,50.36805)(98,51.25925)(99,54.046400000000006)(100,55.760999999999996)(101,70.8968)(102,74.87525)(103,76.40520000000001)(104,90.53165000000001)(105,93.6799)(106,107.758)(107,120.9585)(108,157.3875)(109,197.547)(110,204.172)(111,220.253)(112,305.5425)(113,334.64099999999996)(114,352.45500000000004)(115,352.457)(116,359.1355)(117,387.92150000000004)(118,409.446)(119,412.5855)(120,440.3635)(121,474.86)(122,480.818)(123,489.6075)(124,559.0)(125,559.3405)(126,581.789)(127,591.3720000000001)(128,594.5195)(129,594.7495)(130,595.4855)(131,596.4970000000001)(132,600.0)(133,600.0)(134,600.0)(135,600.0)(136,600.0)(137,600.0)(138,600.0)(139,600.0)(140,600.0)(141,600.0)(142,600.0)(143,600.0)(144,600.0)(145,600.0)(146,600.0)(147,600.0)(148,600.0)(149,600.0)(150,600.0)(151,600.0)(152,600.0)(153,600.0)(154,600.0)(155,600.0)(156,600.0)(157,600.0)(158,600.0)(159,600.0)(160,600.0)(161,600.0)(162,600.0)(163,600.0)(164,600.0)(165,600.0)(166,600.0)(167,600.0)(168,600.0)(169,600.0)(170,600.0)(171,600.0)(172,600.0)(173,600.0)(174,600.0)(175,600.0)(176,600.0)(177,600.0)(178,600.0)(179,600.0)(180,600.0)(181,600.0)(182,600.0)(183,600.0)(184,600.0)(185,600.0)(186,600.0)(187,600.0)(188,600.0)(189,600.0)(190,600.0)(191,600.0)(192,600.0)(193,600.0)(194,600.0)(195,600.0)(196,600.0)(197,600.0)(198,600.0)(199,600.0)(200,600.0)(201,600.0)(202,600.0)(203,600.0)(204,600.0)(205,600.0)(206,600.0)(207,600.0)(208,600.0)(209,600.0)(210,600.0)(211,600.0)(212,600.0)(213,600.0)(214,600.0)(215,600.0)(216,600.0)(217,600.0)(218,600.0)(219,600.0)(220,600.0)(221,600.0)(222,600.0)(223,600.0)(224,600.0)(225,600.0)(226,600.0)(227,600.0)(228,600.0)(229,600.0)(230,600.0)(231,600.0)(232,600.0)(233,600.0)(234,600.0)(235,600.0)(236,600.0)(237,600.0)(238,600.0)(239,600.0)(240,600.0)(241,600.0)(242,600.0)(243,600.0)(244,600.0)(245,600.0)(246,600.0)(247,600.0)(248,600.0)(249,600.0)(250,600.0)(251,600.0)(252,600.0)(253,600.0)(254,600.0)(255,600.0)(256,600.0)(257,600.0)(258,600.0)(259,600.0)(260,600.0)(261,600.0)(262,600.0)(263,600.0)(264,600.0)(265,600.0)(266,600.0)(267,600.0)(268,600.0)(269,600.0)(270,600.0)(271,600.0)(272,600.0)(273,600.0)(274,600.0)(275,600.0)(276,600.0)(277,600.0)(278,600.0)(279,600.0)(280,600.0)(281,600.0)(282,600.0)(283,600.0)(284,600.0)(285,600.0)(286,600.0)(287,600.0)(288,600.0)(289,600.0)(290,600.0)(291,600.0)(292,600.0)(293,600.0)(294,600.0)(295,600.0)(296,600.0)(297,600.0)(298,600.0)(299,600.0)(300,600.0)(301,600.0)(302,600.0)(303,600.0)(304,600.0)(305,600.0)(306,600.0)(307,600.0)(308,600.0)(309,600.0)(310,600.0)(311,600.0)(312,600.0)(313,600.0)(314,600.0)(315,600.0)(316,600.0)(317,600.0)(318,600.0)(319,600.0)(320,600.0)(321,600.0)(322,600.0)(323,600.0)(324,600.0)(325,600.0)(326,600.0)(327,600.0)(328,600.0)(329,600.0)(330,600.0)(331,600.0)(332,600.0)(333,600.0)
};
\addlegendentry{$\texttt{EP-ASP}^{\mathit{se}}$}
      
\addplot[
        color=violet,
        mark=x,
        ]
        coordinates {
        (1,0.031178)(2,0.031935000000000005)(3,0.032090499999999994)(4,0.034354499999999996)(5,0.0345305)(6,0.0345885)(7,0.0353975)(8,0.0355925)(9,0.035702)(10,0.035963999999999996)(11,0.0361375)(12,0.0363715)(13,0.0367045)(14,0.0374455)(15,0.039135500000000004)(16,0.039166)(17,0.040163500000000005)(18,0.0422255)(19,0.051624)(20,0.0525725)(21,0.05848)(22,0.07842450000000001)(23,0.09093899999999999)(24,0.090964)(25,0.09231600000000001)(26,0.09399550000000001)(27,0.10890050000000001)(28,0.125077)(29,0.1549975)(30,0.1679805)(31,0.178454)(32,0.182256)(33,0.1898235)(34,0.206745)(35,0.2478885)(36,0.313266)(37,0.465581)(38,0.503843)(39,0.506104)(40,0.5208124999999999)(41,0.8379369999999999)(42,0.9469165)(43,1.0719699999999999)(44,1.2353999999999998)(45,1.67059)(46,1.777155)(47,2.10996)(48,2.4309000000000003)(49,3.135085)(50,3.759845)(51,4.557295)(52,37.52235)(53,51.25925)(54,101.07650000000001)(55,102.382)(56,119.86949999999999)(57,120.346)(58,162.2465)(59,163.074)(60,169.0085)(61,176.154)(62,187.624)(63,195.91449999999998)(64,198.822)(65,239.502)(66,246.331)(67,267.1975)(68,282.8325)(69,305.5425)(70,313.22)(71,315.86400000000003)(72,334.64099999999996)(73,352.073)(74,352.45500000000004)(75,359.1355)(76,387.92150000000004)(77,396.42949999999996)(78,409.446)(79,412.5855)(80,440.3635)(81,448.235)(82,474.86)(83,480.818)(84,489.6075)(85,556.674)(86,559.0)(87,559.3405)(88,581.789)(89,600.0)(90,600.0)(91,600.0)(92,600.0)(93,600.0)(94,600.0)(95,600.0)(96,600.0)(97,600.0)(98,600.0)(99,600.0)(100,600.0)(101,600.0)(102,600.0)(103,600.0)(104,600.0)(105,600.0)(106,600.0)(107,600.0)(108,600.0)(109,600.0)(110,600.0)(111,600.0)(112,600.0)(113,600.0)(114,600.0)(115,600.0)(116,600.0)(117,600.0)(118,600.0)(119,600.0)(120,600.0)(121,600.0)(122,600.0)(123,600.0)(124,600.0)(125,600.0)(126,600.0)(127,600.0)(128,600.0)(129,600.0)(130,600.0)(131,600.0)(132,600.0)(133,600.0)(134,600.0)(135,600.0)(136,600.0)(137,600.0)(138,600.0)(139,600.0)(140,600.0)(141,600.0)(142,600.0)(143,600.0)(144,600.0)(145,600.0)(146,600.0)(147,600.0)(148,600.0)(149,600.0)(150,600.0)(151,600.0)(152,600.0)(153,600.0)(154,600.0)(155,600.0)(156,600.0)(157,600.0)(158,600.0)(159,600.0)(160,600.0)(161,600.0)(162,600.0)(163,600.0)(164,600.0)(165,600.0)(166,600.0)(167,600.0)(168,600.0)(169,600.0)(170,600.0)(171,600.0)(172,600.0)(173,600.0)(174,600.0)(175,600.0)(176,600.0)(177,600.0)(178,600.0)(179,600.0)(180,600.0)(181,600.0)(182,600.0)(183,600.0)(184,600.0)(185,600.0)(186,600.0)(187,600.0)(188,600.0)(189,600.0)(190,600.0)(191,600.0)(192,600.0)(193,600.0)(194,600.0)(195,600.0)(196,600.0)(197,600.0)(198,600.0)(199,600.0)(200,600.0)(201,600.0)(202,600.0)(203,600.0)(204,600.0)(205,600.0)(206,600.0)(207,600.0)(208,600.0)(209,600.0)(210,600.0)(211,600.0)(212,600.0)(213,600.0)(214,600.0)(215,600.0)(216,600.0)(217,600.0)(218,600.0)(219,600.0)(220,600.0)(221,600.0)(222,600.0)(223,600.0)(224,600.0)(225,600.0)(226,600.0)(227,600.0)(228,600.0)(229,600.0)(230,600.0)(231,600.0)(232,600.0)(233,600.0)(234,600.0)(235,600.0)(236,600.0)(237,600.0)(238,600.0)(239,600.0)(240,600.0)(241,600.0)(242,600.0)(243,600.0)(244,600.0)(245,600.0)(246,600.0)(247,600.0)(248,600.0)(249,600.0)(250,600.0)(251,600.0)(252,600.0)(253,600.0)(254,600.0)(255,600.0)(256,600.0)(257,600.0)(258,600.0)(259,600.0)(260,600.0)(261,600.0)(262,600.0)(263,600.0)(264,600.0)(265,600.0)(266,600.0)(267,600.0)(268,600.0)(269,600.0)(270,600.0)(271,600.0)(272,600.0)(273,600.0)(274,600.0)(275,600.0)(276,600.0)(277,600.0)(278,600.0)(279,600.0)(280,600.0)(281,600.0)(282,600.0)(283,600.0)(284,600.0)(285,600.0)(286,600.0)(287,600.0)(288,600.0)(289,600.0)(290,600.0)(291,600.0)(292,600.0)(293,600.0)(294,600.0)(295,600.0)(296,600.0)(297,600.0)(298,600.0)(299,600.0)(300,600.0)(301,600.0)(302,600.0)(303,600.0)(304,600.0)(305,600.0)(306,600.0)(307,600.0)(308,600.0)(309,600.0)(310,600.0)(311,600.0)(312,600.0)(313,600.0)(314,600.0)(315,600.0)(316,600.0)(317,600.0)(318,600.0)(319,600.0)(320,600.0)(321,600.0)(322,600.0)(323,600.0)(324,600.0)(325,600.0)(326,600.0)(327,600.0)(328,600.0)(329,600.0)(330,600.0)(331,600.0)(332,600.0)(333,600.0)
        };
\addlegendentry{$\texttt{EP-ASP}$}
 \addplot[
    dotted,
    color=orange,
    mark=*,
    ]
    coordinates {
    (1,0.6968365000000001)(2,0.7333505)(3,0.751529)(4,0.772543)(5,0.804659)(6,0.8279460000000001)(7,0.8395475)(8,0.8681105)(9,0.8884495)(10,0.8995615)(11,0.9119785)(12,0.9463535000000001)(13,1.01375)(14,1.064155)(15,1.226655)(16,1.267335)(17,1.33013)(18,1.381665)(19,1.5247000000000002)(20,1.5453800000000002)(21,1.5910199999999999)(22,1.6478199999999998)(23,1.7503199999999999)(24,2.124355)(25,2.25526)(26,2.333435)(27,2.79624)(28,3.266605)(29,4.110595)(30,4.88972)(31,5.20197)(32,5.6643550000000005)(33,5.770004999999999)(34,5.811155)(35,6.875065)(36,7.35468)(37,7.466100000000001)(38,7.501469999999999)(39,7.518275)(40,7.752455)(41,9.400535000000001)(42,10.5714)(43,10.582705)(44,10.936)(45,12.1957)(46,12.30375)(47,14.86505)(48,17.67715)(49,19.5923)(50,22.186700000000002)(51,26.0766)(52,27.9112)(53,30.640549999999998)(54,40.0429)(55,46.2892)(56,47.8786)(57,53.77245)(58,73.4265)(59,78.30824999999999)(60,89.05015)(61,110.4155)(62,120.959)(63,141.584)(64,170.117)(65,223.565)(66,231.8475)(67,240.1805)(68,303.932)(69,350.51099999999997)(70,398.7685)(71,405.9495)(72,532.6395)(73,586.7855)(74,600.0)(75,600.0)(76,600.0)(77,600.0)(78,600.0)(79,600.0)(80,600.0)(81,600.0)(82,600.0)(83,600.0)(84,600.0)(85,600.0)(86,600.0)(87,600.0)(88,600.0)(89,600.0)(90,600.0)(91,600.0)(92,600.0)(93,600.0)(94,600.0)(95,600.0)(96,600.0)(97,600.0)(98,600.0)(99,600.0)(100,600.0)(101,600.0)(102,600.0)(103,600.0)(104,600.0)(105,600.0)(106,600.0)(107,600.0)(108,600.0)(109,600.0)(110,600.0)(111,600.0)(112,600.0)(113,600.0)(114,600.0)(115,600.0)(116,600.0)(117,600.0)(118,600.0)(119,600.0)(120,600.0)(121,600.0)(122,600.0)(123,600.0)(124,600.0)(125,600.0)(126,600.0)(127,600.0)(128,600.0)(129,600.0)(130,600.0)(131,600.0)(132,600.0)(133,600.0)(134,600.0)(135,600.0)(136,600.0)(137,600.0)(138,600.0)(139,600.0)(140,600.0)(141,600.0)(142,600.0)(143,600.0)(144,600.0)(145,600.0)(146,600.0)(147,600.0)(148,600.0)(149,600.0)(150,600.0)(151,600.0)(152,600.0)(153,600.0)(154,600.0)(155,600.0)(156,600.0)(157,600.0)(158,600.0)(159,600.0)(160,600.0)(161,600.0)(162,600.0)(163,600.0)(164,600.0)(165,600.0)(166,600.0)(167,600.0)(168,600.0)(169,600.0)(170,600.0)(171,600.0)(172,600.0)(173,600.0)(174,600.0)(175,600.0)(176,600.0)(177,600.0)(178,600.0)(179,600.0)(180,600.0)(181,600.0)(182,600.0)(183,600.0)(184,600.0)(185,600.0)(186,600.0)(187,600.0)(188,600.0)(189,600.0)(190,600.0)(191,600.0)(192,600.0)(193,600.0)(194,600.0)(195,600.0)(196,600.0)(197,600.0)(198,600.0)(199,600.0)(200,600.0)(201,600.0)(202,600.0)(203,600.0)(204,600.0)(205,600.0)(206,600.0)(207,600.0)(208,600.0)(209,600.0)(210,600.0)(211,600.0)(212,600.0)(213,600.0)(214,600.0)(215,600.0)(216,600.0)(217,600.0)(218,600.0)(219,600.0)(220,600.0)(221,600.0)(222,600.0)(223,600.0)(224,600.0)(225,600.0)(226,600.0)(227,600.0)(228,600.0)(229,600.0)(230,600.0)(231,600.0)(232,600.0)(233,600.0)(234,600.0)(235,600.0)(236,600.0)(237,600.0)(238,600.0)(239,600.0)(240,600.0)(241,600.0)(242,600.0)(243,600.0)(244,600.0)(245,600.0)(246,600.0)(247,600.0)(248,600.0)(249,600.0)(250,600.0)(251,600.0)(252,600.0)(253,600.0)(254,600.0)(255,600.0)(256,600.0)(257,600.0)(258,600.0)(259,600.0)(260,600.0)(261,600.0)(262,600.0)(263,600.0)(264,600.0)(265,600.0)(266,600.0)(267,600.0)(268,600.0)(269,600.0)(270,600.0)(271,600.0)(272,600.0)(273,600.0)(274,600.0)(275,600.0)(276,600.0)(277,600.0)(278,600.0)(279,600.0)(280,600.0)(281,600.0)(282,600.0)(283,600.0)(284,600.0)(285,600.0)(286,600.0)(287,600.0)(288,600.0)(289,600.0)(290,600.0)(291,600.0)(292,600.0)(293,600.0)(294,600.0)(295,600.0)(296,600.0)(297,600.0)(298,600.0)(299,600.0)(300,600.0)(301,600.0)(302,600.0)(303,600.0)(304,600.0)(305,600.0)(306,600.0)(307,600.0)(308,600.0)(309,600.0)(310,600.0)(311,600.0)(312,600.0)(313,600.0)(314,600.0)(315,600.0)(316,600.0)(317,600.0)(318,600.0)(319,600.0)(320,600.0)(321,600.0)(322,600.0)(323,600.0)(324,600.0)(325,600.0)(326,600.0)(327,600.0)(328,600.0)(329,600.0)(330,600.0)(331,600.0)(332,600.0)(333,600.0)
};
\addlegendentry{ \texttt{selp} }
\addplot[
dotted,
color=violet,
mark=x,
]
coordinates {
(1,0.6788974999999999)(2,0.757239)(3,1.3538450000000002)(4,2.242965)(5,2.25536)(6,2.63274)(7,2.647905)(8,2.76502)(9,2.8928700000000003)(10,3.7167149999999998)(11,3.797325)(12,3.831725)(13,4.369875)(14,5.044255)(15,5.52142)(16,5.971285)(17,6.96951)(18,7.680155)(19,9.036275)(20,10.11495)(21,12.04885)(22,17.3519)(23,19.81545)(24,21.11345)(25,24.8537)(26,33.175399999999996)(27,36.95515)(28,263.288)(29,600.0)(30,600.0)(31,600.0)(32,600.0)(33,600.0)(34,600.0)(35,600.0)(36,600.0)(37,600.0)(38,600.0)(39,600.0)(40,600.0)(41,600.0)(42,600.0)(43,600.0)(44,600.0)(45,600.0)(46,600.0)(47,600.0)(48,600.0)(49,600.0)(50,600.0)(51,600.0)(52,600.0)(53,600.0)(54,600.0)(55,600.0)(56,600.0)(57,600.0)(58,600.0)(59,600.0)(60,600.0)(61,600.0)(62,600.0)(63,600.0)(64,600.0)(65,600.0)(66,600.0)(67,600.0)(68,600.0)(69,600.0)(70,600.0)(71,600.0)(72,600.0)(73,600.0)(74,600.0)(75,600.0)(76,600.0)(77,600.0)(78,600.0)(79,600.0)(80,600.0)(81,600.0)(82,600.0)(83,600.0)(84,600.0)(85,600.0)(86,600.0)(87,600.0)(88,600.0)(89,600.0)(90,600.0)(91,600.0)(92,600.0)(93,600.0)(94,600.0)(95,600.0)(96,600.0)(97,600.0)(98,600.0)(99,600.0)(100,600.0)(101,600.0)(102,600.0)(103,600.0)(104,600.0)(105,600.0)(106,600.0)(107,600.0)(108,600.0)(109,600.0)(110,600.0)(111,600.0)(112,600.0)(113,600.0)(114,600.0)(115,600.0)(116,600.0)(117,600.0)(118,600.0)(119,600.0)(120,600.0)(121,600.0)(122,600.0)(123,600.0)(124,600.0)(125,600.0)(126,600.0)(127,600.0)(128,600.0)(129,600.0)(130,600.0)(131,600.0)(132,600.0)(133,600.0)(134,600.0)(135,600.0)(136,600.0)(137,600.0)(138,600.0)(139,600.0)(140,600.0)(141,600.0)(142,600.0)(143,600.0)(144,600.0)(145,600.0)(146,600.0)(147,600.0)(148,600.0)(149,600.0)(150,600.0)(151,600.0)(152,600.0)(153,600.0)(154,600.0)(155,600.0)(156,600.0)(157,600.0)(158,600.0)(159,600.0)(160,600.0)(161,600.0)(162,600.0)(163,600.0)(164,600.0)(165,600.0)(166,600.0)(167,600.0)(168,600.0)(169,600.0)(170,600.0)(171,600.0)(172,600.0)(173,600.0)(174,600.0)(175,600.0)(176,600.0)(177,600.0)(178,600.0)(179,600.0)(180,600.0)(181,600.0)(182,600.0)(183,600.0)(184,600.0)(185,600.0)(186,600.0)(187,600.0)(188,600.0)(189,600.0)(190,600.0)(191,600.0)(192,600.0)(193,600.0)(194,600.0)(195,600.0)(196,600.0)(197,600.0)(198,600.0)(199,600.0)(200,600.0)(201,600.0)(202,600.0)(203,600.0)(204,600.0)(205,600.0)(206,600.0)(207,600.0)(208,600.0)(209,600.0)(210,600.0)(211,600.0)(212,600.0)(213,600.0)(214,600.0)(215,600.0)(216,600.0)(217,600.0)(218,600.0)(219,600.0)(220,600.0)(221,600.0)(222,600.0)(223,600.0)(224,600.0)(225,600.0)(226,600.0)(227,600.0)(228,600.0)(229,600.0)(230,600.0)(231,600.0)(232,600.0)(233,600.0)(234,600.0)(235,600.0)(236,600.0)(237,600.0)(238,600.0)(239,600.0)(240,600.0)(241,600.0)(242,600.0)(243,600.0)(244,600.0)(245,600.0)(246,600.0)(247,600.0)(248,600.0)(249,600.0)(250,600.0)(251,600.0)(252,600.0)(253,600.0)(254,600.0)(255,600.0)(256,600.0)(257,600.0)(258,600.0)(259,600.0)(260,600.0)(261,600.0)(262,600.0)(263,600.0)(264,600.0)(265,600.0)(266,600.0)(267,600.0)(268,600.0)(269,600.0)(270,600.0)(271,600.0)(272,600.0)(273,600.0)(274,600.0)(275,600.0)(276,600.0)(277,600.0)(278,600.0)(279,600.0)(280,600.0)(281,600.0)(282,600.0)(283,600.0)(284,600.0)(285,600.0)(286,600.0)(287,600.0)(288,600.0)(289,600.0)(290,600.0)(291,600.0)(292,600.0)(293,600.0)(294,600.0)(295,600.0)(296,600.0)(297,600.0)(298,600.0)(299,600.0)(300,600.0)(301,600.0)(302,600.0)(303,600.0)(304,600.0)(305,600.0)(306,600.0)(307,600.0)(308,600.0)(309,600.0)(310,600.0)(311,600.0)(312,600.0)(313,600.0)(314,600.0)(315,600.0)(316,600.0)(317,600.0)(318,600.0)(319,600.0)(320,600.0)(321,600.0)(322,600.0)(323,600.0)(324,600.0)(325,600.0)(326,600.0)(327,600.0)(328,600.0)(329,600.0)(330,600.0)(331,600.0)(332,600.0)(333,600.0)
};
\addlegendentry{ elp2qasp }
\end{axis}
\end{tikzpicture}

        \end{column}
    \end{columns}
    % \begin{myitemize}
    %     \item We compare our new solver with~\texttt{EP-ASP},
    %     the previous version of~\texttt{eclingo},
    %     \texttt{selp}, and~\texttt{elp2qasp}.
    % \end{myitemize}
    % \vspace*{-20pt}
\end{frame}
%%%%%%%%%%%%%%%%%%%%%%%%%%%%%%%%%%%%%%%%%%%%%%%%%%%%%%%%%%%%%%%%%%%%%%
% \begin{frame}{Experimental Results (Highlights)}
%     \begin{myitemize}
%         \item Our solver is \speedup{3.3x} faster than the previous version and it solves 87\% more instances.
%         \item It dominates the other solvers in all benchmarks.
%         \begin{itemize}
%             \item In the eligible benchmark, it solves 100\% of the instances and it is \speedup{7.5x} faster than the previous version.
            
%             \item In the Yale benchmark, it solves every instance in less than 1 second (same as the previous version).
%             \begin{itemize}
%                 \item All but \texttt{elp2qasp} solve all instances.
%             \end{itemize}
            
%             \item In the bomb benchmark, it is \speedup{2.4x} faster than~\texttt{EP-ASP} and it solves almost 3 times more instances.
%             \begin{itemize}
%                 \item Compared to the previous version, it is \speedup{2.9x} faster and it solves almost 7.7 times more instances.
%                 \item \texttt{EP-ASP} has a specific mode for conformant planning that uses domain-specific heuristics. When compared to this version, our solver is \speedup{1.5x} faster and solves \speedup{50\%} more instances.
%             \end{itemize}
%         \end{itemize}
%     \end{myitemize}
% \end{frame}
%%%%%%%%%%%%%%%%%%%%%%%%%%%%%%%%%%%%%%%%%%%%%%%%%%%%%%%%%%%%%%%%%%%%%%
\begin{frame}{Conclusions}
    \begin{myitemize}
        \item We have developed a theoretical framework for generate-and-test computation of worldviews of epistemic logic programs and prove its correctness.
        
        \item We have instantiated this framework with a new generator program that propagates the consequences of epistemic decisions.
        \begin{itemize}
            \item We prove that can exponentially reduce the number of candidates while only introducing a linear overhead. 
        \end{itemize}

        \item We implemented a new solver which solves 87\% more instances than state-of-the-art solvers and \speedup{3.3x} faster.
        
        \bigskip
        
        \item \hi{Future work}: we plan to study how to introduce ideas from conflict-driven to further speed up solving epistemic logic programs.
    \end{myitemize}
\end{frame}
%%%%%%%%%%%%%%%%%%%%%%%%%%%%%%%%%%%%%%%%%%%%%%%%%%%%%%%%%%%%%%%%%%%%%%
\frame{\backcoverpage}
%%%%%%%%%%%%%%%%%%%%%%%%%%%%%%%%%%%%%%%%%%%%%%%%%%%%%%%%%%%%%%%%%%%%%%
\end{document}